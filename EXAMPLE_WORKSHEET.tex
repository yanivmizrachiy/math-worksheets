% דוגמה מעשית - דף עבודה במתמטיקה
% העתק את התוכן הזה ל-main.tex כדי לראות דוגמה של דף עבודה אמיתי

\documentclass[a4paper,12pt]{article}
% חבילות עברית
\usepackage{fontspec}
\usepackage[hebrew,english]{babel}
\babelprovide[import, main]{hebrew}
\babelprovide[import]{english}

% פונט חינמי (מותקן ב-Windows)
\setmainfont{Times New Roman}
\newfontfamily\hebrewfont{Times New Roman}

% החבילה המותאמת שלנו
\makeatletter% worksheet.sty - עיצוב דפי עבודה במתמטיקה
% גרסה מתקדמת 2025 - חבילות מודרניות וחזקות ביותר
\NeedsTeXFormat{LaTeX2e}
\ProvidesPackage{worksheet}[2025/12/26 Hebrew Math Worksheet Package - Advanced Edition]

% ============================================
% חבילות בסיסיות - חבילות מודרניות וחזקות
% ============================================

% גרפיקה מתקדמת
\RequirePackage{graphicx}

% מתמטיקה מתקדמת - חבילות חזקות ביותר
\RequirePackage{amsmath,amssymb,amsthm}

% טבלאות מתקדמות
\RequirePackage{array}

% שוליים - הגדרות אופטימליות (אם geometry זמין)
\IfFileExists{geometry.sty}{%
  \RequirePackage{geometry}
  \geometry{
    a4paper,
    top=2cm,
    bottom=2cm,
    left=2cm,
    right=2cm,
    headheight=0.5cm,
    headsep=0.5cm,
    footskip=1cm
  }
}{%
  % אם geometry לא זמין - שימוש בהגדרות ידניות
  \setlength{\topmargin}{-2cm}
  \setlength{\headheight}{0.5cm}
  \setlength{\headsep}{0.5cm}
  \setlength{\textheight}{24cm}
  \setlength{\oddsidemargin}{-2cm}
  \setlength{\textwidth}{17cm}
}

% כותרות - עיצוב מתקדם (אם fancyhdr זמין)
\IfFileExists{fancyhdr.sty}{%
  \RequirePackage{fancyhdr}
  \pagestyle{fancy}
  \fancyhf{}
  \renewcommand{\headrulewidth}{0pt}
  \fancyfoot[C]{\thepage}
}{%
  \pagestyle{plain}
}

% צבעים - תמיכה בצבעים מתקדמת (אם xcolor זמין)
\IfFileExists{xcolor.sty}{%
  \RequirePackage{xcolor}
}{}

% קישורים ו-PDF metadata (אם hyperref זמין)
\IfFileExists{hyperref.sty}{%
  \RequirePackage{hyperref}
  \hypersetup{
    colorlinks=true,
    linkcolor=blue,
    filecolor=magenta,
    urlcolor=cyan,
    pdftitle={דף עבודה במתמטיקה},
    pdfauthor={},
    pdfsubject={מתמטיקה},
    pdfkeywords={מתמטיקה, דף עבודה, עברית}
  }
}{}

% שיפור טיפוגרפיה (אם microtype זמין)
\IfFileExists{microtype.sty}{%
  \RequirePackage{microtype}
}{}

% טבלאות מתקדמות (אם זמינות)
\IfFileExists{booktabs.sty}{%
  \RequirePackage{booktabs}
}{}

\IfFileExists{longtable.sty}{%
  \RequirePackage{longtable}
}{}

\IfFileExists{tabularx.sty}{%
  \RequirePackage{tabularx}
}{}

% כלים נוספים (אם זמינים)
\IfFileExists{enumitem.sty}{%
  \RequirePackage{enumitem}
}{}

\IfFileExists{multicol.sty}{%
  \RequirePackage{multicol}
}{}

% ============================================
% הגדרות מתמטיקה מתקדמות
% ============================================

% משפטים
\theoremstyle{definition}
\newtheorem{question}{שאלה}[section]
\newtheorem*{solution}{פתרון}

% פקודות מתמטיקה מותאמות אישית
\newcommand{\RR}{\mathbb{R}}
\newcommand{\NN}{\mathbb{N}}
\newcommand{\ZZ}{\mathbb{Z}}
\newcommand{\QQ}{\mathbb{Q}}
\newcommand{\CC}{\mathbb{C}}

% ============================================
% הגדרות עיצוב
% ============================================

% שיפור רווחים
\setlength{\parindent}{0pt}
\setlength{\parskip}{0.5em plus 0.1em minus 0.1em}

% רשימות מותאמות (אם enumitem זמין)
\IfFileExists{enumitem.sty}{%
  \setlist[enumerate]{itemsep=0.3em, topsep=0.5em}
  \setlist[itemize]{itemsep=0.3em, topsep=0.5em}
}{}

\endinput
\makeatother

\begin{document}

% כותרת ראשית
\begin{center}
\Large\textbf{דף עבודה - משוואות ליניאריות}\\
\large כיתה: \underline{\hspace{3cm}} \quad שם: \underline{\hspace{4cm}} \quad תאריך: \underline{\hspace{3cm}}
\end{center}

\vspace{1cm}

% שאלה 1
\section*{שאלה 1}
פתרו את המשוואות הבאות:

\begin{enumerate}
\item פתרו: $3x + 7 = 22$
\vspace{2.5cm}

\item פתרו: $5x - 3 = 2x + 9$
\vspace{2.5cm}

\item פתרו: $\frac{x}{2} + 4 = 10$
\vspace{2.5cm}
\end{enumerate}

% שאלה 2
\section*{שאלה 2}
פתרו את מערכת המשוואות הבאה:
\begin{equation}
\begin{aligned}
x + 2y &= 8 \\
3x - y &= 1
\end{aligned}
\end{equation}

\vspace{3cm}

% שאלה 3
\section*{שאלה 3}
נתונה הפונקציה הליניארית $f(x) = 2x + 1$.

\begin{enumerate}
\item חשבו את $f(3)$
\vspace{2cm}

\item מצאו את $x$ כך ש-$f(x) = 7$
\vspace{2cm}

\item האם הנקודה $(2, 5)$ נמצאת על גרף הפונקציה? נמקו.
\vspace{2cm}
\end{enumerate}

% שאלה 4 - גרף
\section*{שאלה 4}
נתונה הפונקציה $y = -x + 3$.

\begin{enumerate}
\item מצאו את נקודת החיתוך עם ציר $y$
\vspace{1.5cm}

\item מצאו את נקודת החיתוך עם ציר $x$
\vspace{1.5cm}

\item ציירו גרף של הפונקציה (השתמשו ב-Desmos או GeoGebra)
\vspace{4cm}
\end{enumerate}

% שאלה 5 - בעיה מילולית
\section*{שאלה 5}
בבית ספר יש 120 תלמידים. מספר הבנות גדול ב-20 ממספר הבנים.

\begin{enumerate}
\item הגדירו משתנה וכתבו משוואה
\vspace{2cm}

\item כמה בנות וכמה בנים יש בבית הספר?
\vspace{2.5cm}
\end{enumerate}

\vspace{1cm}

\begin{center}
\rule{0.5\textwidth}{0.4pt}\\
\textbf{בהצלחה!}
\end{center}

\end{document}

